\documentclass[a4j,proof]{jarticle}
\usepackage[dvipdfmx]{graphicx,color}
\usepackage{algorithm}
\usepackage{algorithmic}
\usepackage{amsthm}
\usepackage{url}
\usepackage{longtable}
\usepackage{amsmath}
\usepackage{bcprules}
\usepackage{proof}
\usepackage{subfigure}

\begin{document}


\section{型判断}
\subsection{式}
式が型付け可能か判断する型判断は、$\Gamma \vdash^{w/r} e:t$の形をしている。
型環境$\Gamma$のもとで$e$を評価すると$e$に$t$という型がつくということを表
している。$w/r$は、$e$に対して書き込みが行われているかどうかを区別するた
めのラベルである。$w$の時は変数に対して書き込みが行われているという意味で
ある。$r$の時、変数の読み込みが行われているという意味である。

\subsection{文}
型判断は
\[
\Theta;\ B;\ C;\ \Gamma \vdash s \Rightarrow \Gamma'
\]
の形で表される。\\
$\Theta$は関数環境、Bはbreak環境、Cはcontinue環境、$\Gamma$, $\Gamma'$は型環境、sはstatementを表している。


\section{型付け規則}
式の型付け規則を図\ref{typing_expr}のように定義する。

\begin{figure}[htbp]
  \infrule[Evar]{
  }
  {
  \Gamma,\ x:t \vdash^{w/r} x:t
  }

  \infrule[Etemp]{
  }
  {
  \Gamma,\ temp:t \vdash^{w/r} \mathtt{temp}: t
  }

  \infrule[Econst]{
  }
  {
  \Gamma \vdash^{w/r} \mathtt{const}:void
  }


  \infrule[Eunop]{
  \Gamma \vdash^{w/r} e:t
  }
  {
  \Gamma \vdash^{w/r} unop\ e:void
  }

  \infrule[Ebinop]{
  \Gamma \vdash^{w/r} e_{1}:t \andalso
  \Gamma \vdash^{w/r} e_{2}:t'
  }
  {
  \Gamma \vdash^{w/r} e_{1}\ binop\ e_{2}:void
  }

  \infrule[Ederef\_r]{
  0 < o
  }
  {
  \Gamma,\ x:\mathtt{pointer}(t,\ o) \vdash^{r} \mathit{*x}:t
  }

  \infrule[Ederef\_w]{
  o = 1
  }
  {
  \Gamma,\ x:\mathtt{pointer}(t,\ o) \vdash^{w} \mathit{*x}:t
  }

  \infrule[Efield]{
  fl \vdash id:t
  }
  {
  \Gamma,\ x:\mathtt{struct}(id,\ fl) \vdash^{w/r} x.f:t
  }
  \caption{式の型付け規則}
  \label{typing_expr}
\end{figure}

文の型付け規則を図\ref{typing_stmt}のように定義する。

\begin{figure}[htbp]
\infrule[Sskip]{
}
{
\Theta;\ B;\ C;\ \Gamma \vdash \mathtt{skip} \Rightarrow \Gamma'
}

%TODO: exprを使った形に書き直す
\infrule[Sassign]{
\Gamma\ \vdash^{w} e_{1}:t_{1} \andalso
\Gamma\ \vdash^{r} e_{2}:t_{2} \andalso
\Gamma'\ \vdash^{w} e_{1}:t_{1}' \andalso
\Gamma'\ \vdash^{r} e_{2}:t_{2}' \\
\mathtt{empty}(t_{1}) \andalso
t_{2} = t_{1}' + t_{2}'
}
{
\Theta;\ B;\ C;\ \Gamma \vdash e_{1} = e_{2} \Rightarrow
\Gamma'
}

\infrule[Scall\_set]{
\Theta(f) = \tilde{t} \rightarrow \tilde{t}' \andalso
\mathtt{empty}(t_{0}) \andalso
\mathtt{return}(f) = t_{1}
}
{
\Theta;\ B;\ C;\ \Gamma,\ \mathtt{temp} : t_{0},\ \tilde{x} : \tilde{\tau} \vdash
\mathtt{temp} = f(\tilde{x}) \Rightarrow
\Gamma,\ t : t_{1},\ \tilde{x} : \tilde{\tau}'
}

\infrule[Scall]{
\Theta(f) = \tilde{t} \rightarrow \tilde{t}'
}
{
\Theta;\ B;\ C;\ \Gamma,\ \tilde{x} : \tilde{t} \vdash f(\tilde{x}) \Rightarrow
\Gamma,\ \tilde{x} : \tilde{t}'
}


\infrule[Sfree]{
o = 1 \andalso o' = 0 \andalso \mathtt{empty}(t)
}
{
\Theta;\ B;\ C;\ \Gamma,\ x:\mathtt{Tpointer}(t,\ o) \vdash \mathtt{free}(x)
\Rightarrow \Gamma,\ x:\mathtt{Tpointer}(t,\ o')
}

\infrule[Smalloc]{
o = 0 \andalso o' = 1 \andalso \mathtt{empty}(t)
}
{
\Theta;\ B;\ C;\ \Gamma,\ \mathtt{temp} : \texttt{Tpointer}(t, o)\ \vdash
\mathtt{temp} = \mathtt{malloc}(e) \Rightarrow
\Gamma,\ \mathtt{temp} : \mathtt{Tpointer}(t, o')
}

%TODO: exprを使った形に書き直す
\infrule[Sassert]{
\Gamma\ \vdash^{r} e_{1}:t_{1} \andalso
\Gamma\ \vdash^{r} e_{2}:t_{2} \andalso
\Gamma'\ \vdash^{r} e_{1}:t_{1}' \andalso
\Gamma'\ \vdash^{r} e_{2}:t_{2}' \\
t_1 + t_2 = t_3 + t_4
}
{
\Theta;\ B;\ C;\ \Gamma \vdash
\mathtt{assert}(e_{1},\ e_{2})
\Rightarrow \Gamma'
}

%TODO: exprを使った形に書き直す
\infrule[Snull]{
\Gamma \vdash^{r} e:t \andalso
\Gamma' \vdash^{r} e:t'
}
{
\Theta;\ B;\ C;\ \Gamma \vdash
\mathtt{assert\_null}(e)
\Rightarrow \Gamma'
}


\infrule[Sseq]{
\Theta;\ B;\ C;\  \Gamma \vdash s_{1} \Rightarrow \Gamma'' \andalso
\Theta;\ B;\ C;\  \Gamma'' \vdash s_{2} \Rightarrow \Gamma'
}
{
\Theta;\ B;\ C;\ \Gamma \vdash s_{1};\ s_{2} \Rightarrow \Gamma'
}

\infrule[Sif]{
\Theta;\ B;\ C:\ \Gamma \vdash s_1 \Rightarrow \Gamma_1 \andalso
\Theta;\ B;\ C;\ \Gamma \vdash s_2 \Rightarrow \Gamma_2 \andalso
\Gamma_1 = \Gamma_2
}
{
\Theta;\ B;\ C;\ \Gamma \vdash \mathtt{if} (e)\ \mathtt{then}\ s_1\ \mathtt{else}\ s_2
\Rightarrow \Gamma_2
}

\infrule[Sswitch]{
\Theta;\ B;\ C;\ \Gamma \vdash s_1 \Rightarrow \Gamma_1 \andalso
\Theta;\ B;\ C;\ \Gamma \vdash s_2 \Rightarrow \Gamma_2 \andalso
\dots \andalso
\Theta;\ C;\ \Gamma \vdash s_m \Rightarrow \Gamma_m \andalso \\
\Gamma' = \Gamma_1 = \Gamma_2 = \dots = \Gamma_m
}
{
\Theta;\ B;\ C;\ \Gamma \vdash
\mathtt{switch}(e)\ \mathtt{case\ n_{1}:}\ s_1;\ \mathtt{case\ n_{2}:}
\ s_2;\ \dots\ \mathtt{case\ n_{m}:}\ s_m;
\Rightarrow \Gamma'
}

\infrule[Sloop]{
\Gamma';\ \Gamma;\ \Gamma \vdash s_1 \Rightarrow \Gamma_1 \andalso
\Gamma';\ \Gamma;\ \Gamma_1 \vdash s_2 \Rightarrow \Gamma_2 \andalso
\Gamma_2 = \Gamma
}
{
\Theta;\ B;\ C;\ \Gamma \vdash \mathtt{loop}(s_1)\ s_2
\Rightarrow \Gamma'
}

\infrule[Sbreak]{
}
{
\Theta;\ \Gamma;\ C;\ \Gamma \vdash \mathtt{break}
\Rightarrow \Gamma'
}

\infrule[Scontinue]{
}
{
\Theta;\ B;\ \Gamma;\ \Gamma \vdash \mathtt{continue}
\Rightarrow \Gamma'
}

%TODO: exprを使った形に書き直す
\infrule[Sreturn]{
\Gamma \vdash e:t \andalso
\Gamma' \vdash e:t' \andalso
t = \mathtt{return}(f) + t'
}
{
\Theta;\ B;\ C;\ \Gamma \vdash \mathtt{return}\ e
\Rightarrow \Gamma
}
\caption{文の型付け規則}
\label{typing_stmt}
\end{figure}

\end{document}