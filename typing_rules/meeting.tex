\documentclass[a4j,proof]{jarticle}
\usepackage[dvipdfmx]{graphicx,color}
\usepackage{algorithm}
\usepackage{algorithmic}
\usepackage{amsthm}
\usepackage{url}
\usepackage{longtable}
\usepackage{amsmath}
\usepackage{bcprules}
\usepackage{proof}
\usepackage{subfigure}

\title{ミーティング記録}
\author{大元 武}
\date{\today}

\begin{document}
\maketitle

\section{研究内容}
C言語を対象とした、メモリリークに関するエラーを検出するための検証器の実装。
OCamlで書かれたC言語のコンパイラCompCertに昨日を追加していく形で実装している。

\section{9/11 進捗状況}
型推論(typing.ml)の部分を実装中。
具体的には、構造体を扱えるように、add\_type(型同士の所有権の足し算)、
eq\_type(型同士の所有権のequality)に Tcomp\_ptr や Tstruct に関するパターンを
追加した。
また、型同士の足し算を表す型Tplus, 型同士が等しいということを表す制約式の型Teqを追加した。

\subsection{add\_type}
構造体に関する以下のパターンを追加
\begin{itemize}
  \item Tcomp\_ptr, Tcomp\_ptr $\rightarrow$ Tplus
  \item Tcomp\_ptr, Tplus $\rightarrow$ Tplus
  \item Tplus, Tplus $\rightarrow$ Tplus
  \item Tcomp\_ptr, t $\rightarrow$ Tcomp\_ptrを一回展開後、再度add\_type
\end{itemize}

\subsection{eq\_type}
構造体に関する以下のパターンを追加
\begin{itemize}
  \item Tcomp\_ptr, Tcomp\_ptr $\rightarrow$ TEq (制約式)
  \item Tcomp\_ptr, t $\rightarrow$ Tcomp\_ptrを一回展開後、再度eq\_type
\end{itemize}
Tplus,Tplus と Tpointer, Tcomp\_ptrのパターンも後々必要?

\subsection{次回 9/15(火) までにやっておきたいこと}
Tcomp\_ptrを一回展開する関数 expand\_comp\_ptr を実装する。
また、展開したcomp\_ptrと展開後の型を保存しておくためのハッシュテーブルを実装する。
これは、一回展開されたcomp\_ptrはそれ以降も同じ型に展開されるようにするためである。

\section{9/15 ミーティング}
TpointerとTcomp\_ptrのadd\_typeで落ちていたので、それに対処する。
Tcomp\_ptrを展開した際に、structをそのまま返すのではなく、Tpointer(t, o, a)として返すようにする。また、struct内の所有権とTcomp\_ptrのintをfreshなものに置き換える。
また、Tcomp\_ptrから所有権を削除していたが、必要になりそうなので元に戻す。

\subsection{次回までにやっておきたいこと}
\begin{itemize}
  \item Tcomp\_ptrに所有権を追加する。
  \item expand\_comp\_ptr内で、structの所有権とcomp\_ptrのintをfreshなものにする。
\end{itemize}

\section{9/18 ミーティング}
Tcomp\_ptrにfreshなintを割り当てる。
freshな所有権を割り当てる時と同じように、一つ関数を作り、Tcomp\_ptrを作るときは
その関数を必ず呼び出すようにする。

\subsection{次回までにやっておきたいこと}
\begin{itemize}
  \item expand\_comp\_ptr内で、structの所有権とcomp\_ptrのintをfreshなものにする。
  \item make\_comp\_ptr (freshなintを割り当てる) を作る
\end{itemize}

\section{10/6 ミーティング}
構造体に関する制約式の解消の前準備として、
必要な情報を表示させるようにprinterを改良する。

\subsection{次回までにやっておきたいこと}
\begin{itemize}
  \item print\_ctypesでTplusの中身を表示するようにする。
  \item print\_constrでTeqの中身を表示するようにする。
  \item comp\_ptrをexpandした結果を保存しているHashtblの中身を表示するようにする。
  \item comp\_ptrにfreshなintを振れてない部分があるので直す。
\end{itemize}

\section{10/16 ミーティング}
今は、rename\_type (OVarの付け替え) の中でcomp\_ptrのintの付け替えも行っていたが、
comp\_ptrのintの付け替えの部分を切り出して他の関数として新しく定義する。

\subsection{次回までにやっておきたいこと}
\begin{itemize}
  \item comp\_ptrのintの付け替えを新しい関数として定義する.
\end{itemize}
\end{document}