\begin{eabstract}

  Programming in the C language requires manual memory management using
  \verb|malloc| and \verb|free|.  Such memory management style is prone
  to bugs such as memory leak and an access to deallocated memory cells.

  Suenaga and Kobayashi proposed a type system for statically verifying
  memory-deallocation safety.  In this type system, a \emph {fractional
  ownership}, auxiliary information that expresses the capability and
  the obligation on a pointer, is assigned to each pointer type.  They
  proved that a well-typed program does not cause bugs related to memory
  deallocation.  They also proposed a type inference algorithm that
  automatically infers a type of each variable and each function and
  implemented a verifier based on the algorithm.  The type inference
  algorithm generates constraints that the ownerships should satisfy and
  checks the satisfiability of the constraints.

  However, their type system is designed only for a subset of the C
  language; it does not handle some statements such as control
  statements. Although their verifier supports the control statements,
  they support these statements in such a way that is not strictly based
  on the formally defined type inference algorithm.

  %Therefore, their verifier for C programs is not strictly based on the
  %formally defined type inference algorithm.

  %Their verifier can handle all statements of the C language.  However,
  %their type system is designed only for a subset of the C language
  %; it does not handle some statements of the C language such as control
  %statements. Therefore, their verifier for C programs is not strictly
  %based on the formally defined type inference algorithm.


  Our aim is to extend their framework so that it deals with the C
  language in a more dependable way.  Concretely, we extend (1) their
  language with the control statements and (2) the type system to handle
  these statements. We implement a verifier based on the extended type
  system.  Our implementation supports the control statements, data
  structures such as singly linked lists, and definitions of mutually
  recursive functions.

  In order to make the implementation more useful, we incorporate a
  \emph{type-error slicer} to our verifier.  If a program is not
  well-typed, the verifier displays the line numbers of the commands in
  the program that are involved in the type error. This feature is
  useful for finding the cause of the type error.

  We conducted preliminary experiments with the implementation.  We
  confirmed that our implementation correctly detects memory leak of
  programs that include the control statements, that manipulate
  recursive data structures, and that include mutually recursive
  functions.

\end{eabstract}