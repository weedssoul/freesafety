\subsection{制約解消}
制約式変換を終えた制約式は,SMTソルバで解ける形になっている.その制約式を
SMTソルバで解くことで,制約式が充足可能かどうか判定することができる.今回
の実装ではSMTソルバには,Z3 \cite{DBLP:conf/tacas/MouraB08}を使用した.ソ
ルバが制約を充足可能と判定した場合は,プログラムに型がつき,メモリ操作に
関する誤りが発生していないことがわかる.ソルバが充足不能と判定した場合は,
同時にunsat coreを返す.unsat coreとは,制約式の充足不能性の原因となって
いる制約式の部分集合である.unsatの場合は,プログラム中のどこかでメモリ操
作に関する誤りがあり,その箇所が型エラーの原因となっている.その箇所を特
定するのは,小さなプログラムでは容易だが,大きなプログラムだと困難である.
そこで,unsat coreを用いて型エラーの原因となっている命令の集合をプログラ
ム中から切り出して,表示する.この情報は,プログラマが型エラーの原因を特
定するのに役立つと考えられる.

\subsection{型エラースライサー}
型エラースライサーは,制約式の集合とunsat coreを入力として,型エラーの原
因に関係している命令の行番号をスライスとして出力する.unsat coreには制約
式の充足不能性の原因,つまり型エラーの原因となっている制約式が含まれてい
る.そこで,unsat coreに含まれている各制約式がそれぞれどの命令から生成さ
れたものなのかを調べる.制約式には,その制約式を生成した命令の行番号
$loc$が含まれているので,unsat coreに含まれている各制約式の$loc$をまとめ
たものをスライスとすればよい.

今回ソルバとして使用したZ3は,unsat coreを制約式の名前で返す.そのため,
Z3に制約式を渡す際に,各制約式に名前をつけ,どの制約式にどういう名前をつ
けたのかを記録しておくことで,Z3がunsat coreとして返してきた制約式の名前
から制約式を特定し,$loc$を調べることができる.

\begin{example}[型エラースライサー]
  プログラム \ref{rec_struct_unsat}\ (付録参照)\ を例に挙げる.このプログラム
  は,再帰を含む構造体\texttt{list}を使って単方向リストの操作を行っている
  プログラムである.\texttt{make\_list}で単方向リストの各要素に新しいメモ
  リ領域を割り当て,\texttt{free\_all\_list}でそれら全てを解放する.しか
  し,\texttt{free\_all\_list}内で,\texttt{free}を実行し忘れているため,
  メモリリークが発生している.

  このプログラムから制約式を生成し,Z3で解くと
\begin{verbatim}
    unsat
    (c_70 c_67 c_66 c_64 c_54 c_52
     c_49 c_48 c_32 c_31 c_8 c_3 c_7)
\end{verbatim}
  を返す.メモリリークが発生しているため制約式は充足不能である.
  また充足不能のため,unsat coreを一緒に返している.
  unsat coreの各要素は,制約式の名前である.
  型エラースライサーは,これらの名前がついた制約式を生成した命令を探し
  それの行番号をスライスとして出力する.
\begin{verbatim}
    [11; 18; 30; 32; 39; 40; 43]
\end{verbatim}

  $30$行目の\texttt{malloc}で新しいメモリ領域を
  割当て,$32$行目の\texttt{return}でその変数を返り値としているため,$39$行
  目で\texttt{make\_list}を呼び出しその返り値を代入している変数\texttt{l}に
  は,所有権が与えられているはずである.しかし,$40$行目で
  \texttt{free\_all\_list}を呼び出しているにもかかわらず,$43$行目で,関数
  の終了時には関数の局所変数の所有権,つまり\texttt{l}内の所有権は全て$0$に
  なっていないといけないという条件をみたすことができていない.更に$18$行目
  の\texttt{free\_all\_list}もスライスに含まれていることから,
  \texttt{free\_all\_list}の実装が間違っているのではないかと,プログラマは
  推測することができる.
\end{example}