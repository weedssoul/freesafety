\subsection{CompCert}
CompCert \cite {DBLP:journals/cacm/Leroy09}は,Leroyらによって実装された
C言語のコンパイラである.CompCertの最大の特徴は,コンパイラの正しさを数学
的に証明をしている,という点である.コンパイラは,コンパイル時に多くの変
換を行う.しかし,それらの変換が正しく行われているかどうかは,テストによっ
てのみ保証されている.テストによる保証には限界があるため,Leroyらは,言語
に意味論を与え,変換を行った際,変換後の言語が変換前の言語の意味を正確に
保存しているというという性質を,証明支援系Coq
\cite{DBLP:series/txtcs/BertotC04} を用いて証明した.また,Coqには
ExtractionというCoqで書かれたコードを他の言語に変換する機能がある.この機
能により,証明された変換をそのままプログラムにすることができる.CompCert
は,このExtractionによって変換されたOCamlのコードが大部分を占めている.
\label{compcert}

\subsection{Clight}
Clight \cite {DBLP:journals/jar/BlazyL09} は,CompCert内で使われている中
間言語で,C言語のサブセットである.Clightは,ポインタや配列や構造体などの
データ構造,$\texttt{if}$や$\texttt{switch}$などの分岐,$\texttt{while}$
や$\texttt{for}$などの再帰,$\texttt{break}$や$\texttt{continue}$などの構
文,再帰関数や関数ポインタなどの関数に関する機能など,C言語の機能をほぼ全
てサポートしているが,以下の機能は排除されている.

\begin{itemize}
  \item $\texttt{long\ long}$ や $\texttt{long\ double}$ などの拡張された数字
  \item 可変長引数の関数
  \item 式内で副作用を起こす演算子
  \item ブロック内での変数宣言
  \item unstructured switch
\end{itemize}

式内で副作用を起こす演算子とは,インクリメント文などのことである.インクリ
メントは,$x = x + 1$のように文で表現される.また,C言語では,
$\texttt{if}$や$\texttt{while}$などのブロックの中でも変数を宣言できるが,
Clightでは,変数は大域変数か関数のローカル変数だけである.

\texttt{switch}文は,変数の値に応じて処理を分岐させる構文である.
\begin{verbatim}
    switch (x) {
      case 1: s1;
      case 2: s2;
      case 3: s3;
      default: s0;
    }
\end{verbatim}
一般の\texttt{switch}文は$x$の値に応じて1つだけcase文が実行される.
しかし,C言語の場合,s1,s2,s3に\texttt{break}を書かないと,それ以降全ての
case文を実行する.例えば$x$の値が$2$だったとすると,case 2と case 3が実行される.
このような\texttt{switch}のことを,unstructured switch と呼ぶ.
Clightでは,このunstructured switch は扱えず,全てのcase文に\texttt{break}が
挿入されているものとみなされる.

\label{clight}