\subsection{背景}
C 言語\cite{DBLP:books/ph/KernighanR78}は,1972年に Dennis M. Ritchie ら
によって開発されたプログラミング言語である.本来,ソフトウェアは特定のハー
ドウェア上で動作するように設計されている.そのため,他のハードウェア上で
そのソフトウェアを動作させるためには,ソフトウェアを書き換えたり修正した
りする必要がある.この書き換えや修正のことを移植と呼ぶ.C言語は,当時開発
されていた OS である UNIX の移植性を高めるために開発された.OS を書くため
に書かれた言語のためハードウェアよりの低水準な記述もできる,C 言語のコン
パイラ自体の移植性や拡張性も高い,C 言語で記述された UNIX が広く普及した,
などの理由から今でも多くの分野で使用されているプログラミング言語である.

C 言語は上で述べたように古くに開発された言語のため,自動でメモリ管理をす
る仕組みなどは組み込まれていない.そのため,プログラマが手動でメモリ領域
の確保と解放を行う必要がある.C言語におけるメモリ確保は,静的確保と動的確
保の2種類ある.静的確保は,プログラム中で確保するメモリサイズを指定する方
法である.しかし,この手法は予め使用するメモリサイズを予測して確保するた
め,無駄に多くメモリ領域を確保してしまったり,確保したメモリ領域が足りな
いなどの問題が発生する.動的確保は,この問題点を改善したもので,プログラ
ムの実行中に必要な分だけメモリ領域を確保する方法である.この動的確保に使
う命令が\verb|malloc|である.\verb|malloc|は確保するメモリサイズのバイト
数を引数として取り,指定されたサイズのメモリ領域を確保し,先頭のアドレス
を返す.そのアドレスをポインタに代入すれば,そのポインタを通してメモリ領
域に対して操作を行うことができる.確保したメモリ領域を使い終わった後は,
そのメモリ領域を解放する必要がある.このメモリ領域の開放に使う命令が
\verb|free|である.\verb|free|は,ポインタを引数として取り,そのポインタ
の参照先のメモリ領域を解放する.


C 言語ではこれらの操作をプログラマが手動で行うため,種々の誤りが発生しう
る.1つ目は\emph{メモリリーク}である.これは,\verb|malloc|によって確保し
たメモリ領域を解放し忘れるという誤りで,動的確保したメモリ領域を解放し忘
れるとシステムが新たに確保できるメモリ領域がどんどん少なくなってしまい,
その結果システム全体が停止してしまう,という状況を引き起こす.2つ目は
\emph{ダブルフリー}である.これは,一度解放したメモリ領域をもう一度解放し
てしまうというものである.これにより,メモリ管理情報の一貫性が破壊される
ことがある.他には,一度解放したメモリ領域から読み込みを行ったり,そこへ
書き込みを行うなどがある.一度解放したメモリ領域の値は未定義となっており,
そこに操作を行うと予期せぬ動作を起こす原因となる.このようにメモリ操作に
関する誤りは深刻なエラーを引き起こしてしまう.更に,プログラムが大きくな
るに連れてメモリ操作の誤りを発見することが難しくなってしまう.

そこで,SuenagaとKobayashiはプログラム中のメモリ操作に関する誤りを静的に
検出するための型システム\cite {DBLP:conf/aplas/SuenagaK09}を提案した.こ
の型システムでは,所有権と呼ばれる$0$以上$1$以下の有理数でポインタ型を拡
張している.例えば,所有権で拡張された型は$\texttt{int\ ref}_{f}$のように
表される.この型はポインタについている所有権が$f$で,このポインタを一回参
照した先の型が\texttt{int}である,ということを表してる.この所有権の値に
応じてポインタを通してプログラマが行ってよい操作や,行わなければならない
操作を定義している.所有権の値が0の時は,プログラマはポインタを通してどの
ような操作も行うことができない.所有権が$0$より大きく,$1$未満の時は,プ
ログラマはポインタを通して読み込みだけ行うことができる.所有権の値が$1$の
時は,プログラマはポインタを通して読み込み,書き込み,解放を行うことがで
きる.また所有権の値が$0$より大きい時は,プログラマはそのポインタの参照先
のメモリ領域を解放しなければならない.このポインタ型を使い,プログラムに
型がつけばプログラム中でメモリ操作に関する誤りが起きないということが数学
的に証明されている.また彼らは型システムに基いて,各変数や各関数にどのよ
うな型がつくのかを自動で推論する型推論アルゴリズムを提案した.型推論は,
型付け規則に基いて各命令から所有権が満たすべき制約式を生成し,その制約式
の充足可能問題に帰着させている.制約式が充足可能であればプログラムに型が
つくということがわかる.

\label{background}

\subsection{本論文の目的}
彼らは,提案した型推論アルゴリズムに基いて検証器を実装した.しかし,彼ら
が論文中で提案した型システムは\verb|for|,\verb|while|,\verb|break|,
\verb|continue|などの制御文に対して形式的な定義は与えていない.そのため,
検証器の制御文を扱っている部分は,厳密に理論に基づいた実装になっていない.
そこで本研究では,より信頼度の高い方法で検証器の実装を行う.具体的には,
彼らが提案した型システムで扱う言語を制御文で拡張し,拡張した言語に対して
型付け規則を与える.そして拡張した型システムに基いて検証器の実装を行う.
この検証器は制御文だけでなく,単方向リストなど再帰を含むデータ構造や,相
互再帰関数なども扱うことができる.更に,検証器の利便性を高めるために型エ
ラースライサーを検証器に組み込んだ.型エラースライサーはプログラムに型が
つかないとわかった時,つまり制約式が充足不能とわかった時,充足不能の原因
となっている制約式の部分集合である unsat coreを用いて,型エラーの原因となっ
ている命令をスライスとして表示する.この情報は,プログラマがプログラム中
のメモリ操作の誤りを発見するにの役立つと考えられる.

\label{object}

\subsection{本論文の構成}
本論文の構成は以下の通りである.第2章では,SuenagaとKobayashiによって提案
された,メモリ操作に関する誤りを静的に検出するための型システムと型推論ア
ルゴリズムについて述べる.第3章では,本研究で検証の対象とする言語 Clight
\cite{DBLP:journals/jar/BlazyL09}について説明をし,提案された型システムを
Clightへ拡張する.第4章では,拡張された型システムに基づいた検証器の実装に
ついて述べる.第5章では,実装した検証器に対して行った予備実験について述
べる.第6章では,関連研究の紹介と本研究との比較を行う.第6章では,本研究
のまとめと今後の課題について述べる.

\label{paper}