\begin{jabstract}
  C 言語では,プログラマは\verb|malloc|や\verb|free|などの命令を用いて,
  手動でメモリ管理を行わなければならない.そのため,メモリリークや,一度
  解放したメモリ領域へのアクセスなどの,メモリ管理上の誤りを起こすことが
  ある.

  SuenagaとKobayashiは,これらの誤りを静的に検出するための型システムを提
  案した.この型システムでは,ポインタを表す型を,\emph{所有権}と呼ばれる
  情報で拡張している.この情報は,プログラマがそのポインタを通してどのよ
  うな操作を行ってよいか,行わなければならないかを表している.プログラム
  に型がつけば,メモリ管理に関する誤りが起きないということが証明されてい
  る.彼らはこの型システムに基づいて,各変数や各関数にどのような型がつく
  のか自動で推論する型推論アルゴリズムを提案し,そのアルゴリズムに基づき
  検証器を実装した.型推論は,各命令から型付け規則に基づき,所有権が満た
  すべき制約式を生成し,その制約式の充足可能問題に帰着させている.

  しかし,彼らの型システムはC言語のサブセットを対象としているため,制御文
  など,いくつかの構文を扱うことができない.彼らの検証器は制御文も扱うこ
  とができるが,その部分の実装は厳密には理論に基づいたものにないっていな
  い.

  %彼らの検証器は,C 言語の全ての構文を扱うことができる.しかし,彼らの型
  %システムは,C言語のサブセットを対象としているため,制御文などいくつかの
  %構文は扱うことができない。そのため,彼らの検証器は,型推論アルゴリズム
  %に厳密に基づいたものになっていない。


  そこで本研究では,より信頼度の高い方法で検証器の実装を行う.具体的には,
  (1) 制御文で言語を拡張し,(2) その構文を扱えるように型システムを拡張す
  る.また,拡張した型システムに基づいて実装を行った.この検証器は,制御
  文に加えて,単方向リストなどのデータ構造や,相互再帰関数も扱うことがで
  きる.

  更に実装の利便性を高めるために\emph{型エラースライサー}を検証器に組み込
  んだ.型エラーが起こった場合,検証器は型エラーに関係する命令の行番号を
  表示する.この情報は,型エラーの原因を特定するのに役立つ.

  また,実装した検証器に対して予備実験を行った.検証器が制御文を含むプロ
  グラム,再帰のある構造体を扱ったプログラム,相互再帰関数を含むプログラ
  ムに対しても,メモリリークを正しく検出できることを確かめた.
\end{jabstract}