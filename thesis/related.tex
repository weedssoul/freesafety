本研究では,末永らが提案したメモリリークを静的に検出する型システム
\cite{DBLP:conf/aplas/SuenagaK09}を制御文で拡張しした.また,その拡張した
型システムに基いて検証器を実装した.検証器の実装においては,Leoryらによっ
て実装されたC言語のコンパイラCompCert
\cite{DBLP:journals/cacm/Leroy09,DBLP:conf/itp/KrebbersLW14 }を拡張する形
で実装した.また型システムの拡張も,CompCertで使われている中間言語
Clight \cite{DBLP:journals/jar/BlazyL09} を元にしている.

CompCertを使用した他の研究としては\cite{DBLP:conf/popl/StewartBCA15}など
がある.従来のCompCertは他のモジュール内の関数の呼び出しの正しさはは保証
しない.そこで,この研究は他のモジュール内の関数の呼び出しなどの正しさも
保証するComposite CompCertを提案している.

本研究で,SMTソルバとして使用した Z3 を利用した研究として
\cite{DBLP:conf/fmcad/McMillan11}などがある.これは,Z3を利用して,モデル
検査の分野で使用される補間を求める手法を提案している.

型エラースライサーは,静的型付けの行われている関数型言語に対して,型エラー
が発生した際に,型エラーに関係しているプログラム地点をスライスとして出力
するツールである\cite{DBLP:conf/esop/HaackW03,DBLP:conf/dsl/DineshT97}.
更に,型エラースライサーは本来,関数型言語を対象としていたが,近年Javaな
どの言語に対しても研究が行われている\cite{DBLP:conf/pepm/BoustaniH09}.一
般の静的型付け言語において,型推論はプログラムを抽象構文木に変換した後,
各ノードに対して型付け規則に基いて型がつくかどうか確認していき,失敗した
場合,失敗したノードを1つだけ型エラーの原因として出力する.しかし,型エラー
の原因が1つしか挙げられないため,プログラマは型エラーの原因を正確に特定す
ることができない.そこで型エラースライサーは,各ノードから生成された制約
式にラベルをつけ,制約式全体が充足不能とわかった場合,充足不能となってい
る最小の制約式の集合を抽出し,その集合に含まれている制約式のラベルからプ
ログラム地点を特定し,スライスとして出力するということをしている.本研究
の型エラースライサーもこの考えに基づいて実装をしている.\ref{section4}で
述べたように,制約式の充足不能の原因を抽出する際に,SMTソルバが返す
unsat core を用いている.また,これまでの型エラースライサーは単純型を対象
としていたが,本研究では,所有権型を対象としている.\cite{飯村枝里2007,
飯村枝里2008}では,並行プログラミングにおけるレースの原因やデッドロックの
原因を型エラースライサーを用いて特定しているが,制約式の充足不能の原因特
定に unsat core は用いられていない.

他のメモリリークの検証器は,XieらによるSATソルバを用いた
\cite{DBLP:conf/sigsoft/XieA05}や,Cheremらによるグラフを用いた
\cite{DBLP:conf/pldi/CheremPR07}などがある.しかし,両方ともループを健全
な形で扱うことができない.

Hoare 論理を拡張して,領域の確保や解放,領域へのアクセスなどメモリに関す
る操作をを扱えるようにした体系としてSeparation Logic
\cite{DBLP:conf/lics/Reynolds02}が提案されている.この Separation Logic
に基いて,メモリリークの検出やエイリアス解析などを行う手法として Shape
analysis
\cite{DBLP:conf/cav/BerdineCCDOWY07,DBLP:conf/tacas/DistefanoOY06,
DBLP:conf/pldi/GuoVA07} という手法がある.しかし,これらの手法は検出に非
常に時間がかってしまう.