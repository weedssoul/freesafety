本研究では,末永らが提案したメモリリークを静的に検出するための型システム
を制御文で拡張し,その拡張した型システムに基いて検証器を実装することで,
検証器の信頼度を高めた.また,型エラースライサーを組み込むことで,検証器
の利便性も高めた.更に,予備実験を行いいくつかのプログラムに対して,メモ
リリークを正しく検出できることを確かめた.

今後の課題として,まず扱える構文を増やすことが挙げられる.\ref{section3}
で述べたように,配列型,キャストやアドレス演算子などの式,\texttt{goto}や
\texttt{label}などの文,大域変数は,本研究で拡張した型システムではまだ扱っ
ていない.\texttt{goto}や\texttt{label}などは,\texttt{break}や
\texttt{continue}の考えを応用すれば扱えるようになると考えられる.また,拡
張した型システムの健全性の証明を行っていない.元の型システムでは,プログ
ラムに型がつけば,プログラム中でメモリリークが起きてないことが保証されて
いたが,今回拡張した制御文などの構文に関しては,その性質の証明ができてい
ない.更に,より大きなプログラムに対して実験を行うことが挙げられる.今回
の予備実験ではいずれも小さなプログラムに対してのみ実験を行ったので,より
大きなプログラムでもメモリリークを正しく検出できるか,どの程度早く検出で
きるかなどを確かめる必要がある.同時に,型エラースライサーが型エラーの原
因の特定にどの程度有効なのかを確認し,改良していく必要がある.