この章では,本研究で実装した検証器に対して行った予備実験について述べる.

\begin{table}[htbp]
  \footnotesize
  \centering
  \caption{実験結果}
  \label{ex_table}
  \begin{tabular}[tb]{|l|r|r|r|r|r|r|}
    \hline
    Program & LOC & Time\_g & SIZE\_g & Time\_r & SIZE\_r & Time\_s \\ \hline\hline
    \verb|sl_app| & 55 & 0.165 & 200 & 0.748 & 193 &  34.0 \\ \hline
    \verb|sl_free| & 44 & 0.112 & 114 & 0.505 & 110 & 26.5 \\ \hline
    \verb|sl_merge| & 60 & 0.280 & 250 & 1.395 & 237 & 37.5 \\ \hline
    \verb|sl_mut| & 60 & 0.140 & 149 & 0.495 & 145 & 30.9 \\ \hline
    \verb|sl_reverse| & 63 & 0.173 & 222 & 0.901 & 205 & 33.2 \\ \hline
    \verb|sl_search| & 60 & 0.189 & 189 & 0.654 & 176 & 36.8 \\ \hline
  \end{tabular}
\end{table}

表\ref{ex_table}は実験結果をまとめたものである.
各列の意味は以下のとおりである.
\begin{itemize}
  \item LOC: プログラムの行数
  \item Time\_g: 制約生成にかかった時間(msec)
  \item SIZE\_g: 生成された制約式の数
  \item Time\_r: 制約変換にかかった時間(msec)
  \item SIZE\_g: 変換後の制約式の数
  \item Time\_s: 制約解消にかかった時間(msec)
\end{itemize}

実験に用いた計算環境は CPU: Intel Core i7 1.7 GHz,MEM:8GB,OS:OSX
10.9.5 である.使用したソフトの各バージョンは,CompCert: 2.4, OCaml:
4.02.1, Z3: 4.3.2 である.

いずれのプログラムも,単方向リストを操作するプログラムで,末永らが先行研
究で実装した検証器に対して行った予備実験で使われたプログラムと同等のもの
である.各プログラムには,$\texttt{assert\_null}(p)$という注釈を手動で適
当な箇所に挿入してある.これは,\ref{section3}で述べたように,ポインタの
参照先を\texttt{null}とみなして,任意の型をもてるようにするための命令であ
る.

\verb|sl_app|は,リストを2つ生成した後,片方のリストをもう片方のリストの
最後尾に繋げたリストを生成し,それを繋げたリストを生成し,そのリストを解
放する.\verb|sl_free|は,単純にリストを生成しそれを解放する.
\verb|sl_merge|は,リストを2つ生成した後,それらのリストの先頭から要素を
ランダムに選んで繋げたリストを生成し,そのリストを解放する.
\verb|sl_mut|は,\verb|sl_free|と同じ操作を行っているが,相互再帰関数を使っ
て解放を行っている.\verb|sl_reverse|は,リストを生成した後,そのリストの
順番を逆にし,そのリストを解放する.\verb|sl_search|は,リストを生成した
後,リストから要素を一つ取り出してその要素を除いたリストを生成し,取り出
した要素とそれを除いたリストを解放する.

いずれのプログラムもサイズは大きくないが,非常に高速に検出することができ
ている.生成された制約式が変換によって数が減っているのは,変換の際に生成
された制約式の中に,同じ制約式があるとそれを消す操作を行っているためであ
る.

また,各プログラムに対してメモリリークが発生するように,\texttt{free}命
令を消すなどの書き換えを行い,実験を行った.結果として,全てのプログラムに対
して検証器は充足不能と判定し,エラーを正しく検出していることを確認した.